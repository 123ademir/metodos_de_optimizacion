\documentclass[12pt]{article}
\usepackage[spanish]{babel}
\usepackage[utf8]{inputenc}
\usepackage[T1]{fontenc}
\usepackage{amsmath}
\usepackage{geometry}
\usepackage{hyperref}
\usepackage{booktabs}
\usepackage{array}
\geometry{margin=2.5cm}

\title{\textbf{Variable, Funci\'on y Restricci\'on} \\ \large \textbf{Pobreza en el Per\'u en 2020}}
\author{\textbf{Ing.:} Torres Cruz Fred \\ \textbf{Estudiante:} Ruelas Yana Nestor Ademir \\ \textbf{C\'odigo:} 230868 \\ \textbf{Curso:} M\'etodos de Optimizaci\'on}
\date{12 de abril de 2025}

\begin{document}

\maketitle

\section*{1. Introducci\'on}
La pobreza es un desafío que impacta a numerosas familias en nuestro país. Este informe se enfoca en la pobreza monetaria, utilizando datos oficiales del Perú durante el año 2020, un periodo especialmente complejo debido a la pandemia, que provocó una considerable pérdida de ingresos en la población.

\section*{2. \textbf{¿Qu\'e es una variable?}}
Una variable representa un elemento que puede cuantificarse o modificarse. En este análisis se emplea la variable \textbf{pobreza monetaria}, la cual indica la proporción de personas que carecen de recursos económicos suficientes para cubrir necesidades básicas como la alimentación, el vestido y servicios esenciales.

\subsection*{Dato real}
De acuerdo con el INEI, en el año 2020 el 30.1\% de los peruanos se encontraba en condición de pobreza.

\begin{quote}
\textit{Fuente: INEI (2021). Evolución de la pobreza monetaria 2020. Informe Técnico. Lima.}
\end{quote}

\subsection*{Cuadro: Evolución de la pobreza en el Perú (2019--2020)}
\begin{center}
\begin{tabular}{|>{\raggedright}m{5cm}|c|c|}
\hline
\textbf{Indicador} & \textbf{2019} & \textbf{2020} \\
\hline
Pobreza total (\%) & 20.2 & 30.1 \\
\hline
Pobreza urbana (\%) & 14.6 & 26.0 \\
\hline
Pobreza rural (\%) & 40.8 & 45.7 \\
\hline
\end{tabular}
\end{center}

\textit{Fuente: INEI, Informe Técnico, Gráficos (3.1 - 3.2).}

\section*{3. Organizaci\'on de la variable}
\begin{itemize}
  \item \textbf{Nombre:} Pobreza monetaria
  \item \textbf{Tipo:} Numérica (expresada en porcentajes)
  \item \textbf{Unidad de análisis:} Individuos o hogares
  \item \textbf{Fuente de datos:} Encuesta Nacional de Hogares (ENAHO)
\end{itemize}

\section*{4. C\'omo se mide (operacionalizaci\'on)}
\begin{itemize}
  \item \textbf{Indicador:} Porcentaje de personas cuyos ingresos son inferiores al valor de una canasta básica de consumo
  \item \textbf{Recolección de datos:} A través de encuestas aplicadas por el INEI
  \item \textbf{Procedimiento de cálculo:} Se comparan los ingresos mensuales del hogar con el costo de cubrir las necesidades mínimas
\end{itemize}

\section*{5. Finalidad de la variable (funci\'on)}
Esta variable permite identificar a la población que enfrenta dificultades económicas severas. Es una herramienta útil para que las autoridades y organizaciones diseñen políticas públicas y programas de ayuda más eficaces y focalizados.

\section*{6. Limitaciones al medirla (restricciones)}
\begin{itemize}
  \item Algunas personas no reportan con precisión sus ingresos
  \item Los precios de los bienes varían entre regiones
  \item En ciertas zonas es difícil aplicar encuestas de manera eficiente
\end{itemize}

\section*{7. Conclusi\'on}
Contar con información sobre los niveles de pobreza es clave para comprender la realidad socioeconómica del país. Su análisis permite orientar esfuerzos hacia quienes más lo requieren.

\section*{8. Referencias}
\begin{itemize}
  \item Instituto Nacional de Estadística e Informática (INEI). (2021). \textit{Evolución de la pobreza monetaria 2020}. Lima: INEI. Recuperado de: \url{https://www.inei.gob.pe/}
  \item Hernández, R., Fernández, C. y Baptista, P. (2014). \textit{Metodología de la investigación}. McGraw-Hill.
\end{itemize}

\end{document}

